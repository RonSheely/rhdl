\documentclass[conference]{IEEEtran}
\IEEEoverridecommandlockouts
% The preceding line is only needed to identify funding in the first footnote. If that is unneeded, please comment it out.
\usepackage{cite}
\usepackage{amsmath,amssymb,amsfonts}
\usepackage{algorithmic}
\usepackage{graphicx}
\usepackage{textcomp}
\usepackage{xcolor}
\usepackage{minted}
\def\BibTeX{{\rm B\kern-.05em{\sc i\kern-.025em b}\kern-.08em
    T\kern-.1667em\lower.7ex\hbox{E}\kern-.125emX}}
\begin{document}

\title{RustHDL and RHDL - two ongoing experiments in using Rust as a Hardware Description Language}

\author{\IEEEauthorblockN{Samit Basu}
  Fremont, California
  USA
  basu.samit@gmail.com
}

\maketitle

\begin{abstract}
  RustHDL and RHDL are two ongoing experiments in the use of the Rust programming language for the
  purposes of hardware description.  Both are open source projects that aim to bring the benefits
  of the Rust programming language to the task of firmware design, and focus on enabling the use
  of features such as strong typing, ease of design reuse, and rigorous safety and linting in
  the implementation of firmware.  RustHDL is the older of the two projects, has a fairly extensive
  library of IP cores for common hardware tasks, and has been fielded in commercial systems.
  RHDL is a next generation version of RustHDL that attempts to address some of its shortcomings.
  This paper attempts to describe both along with the rationale for the rewrite.
\end{abstract}

\begin{IEEEkeywords}
  Hardware description languages, Rust Programming Language, Field Programmable Gate Arrays,
  Design automation
\end{IEEEkeywords}

\section{Introduction}
There are a number of new hardware description languages (HDL) that are being introduced that
attempt to remedy some of the shortcomings of the traditional Verilog or VHDL based workflow.
These HDLs tend to focus on newer features borrowed from the rapidly advancing fields of
software programming language design and compiler development.  Typically, these new HDLs
come with custom toolchains, new syntaxes and grammers, as well as new ways of thinking about
how hardware designs should best be expressed in text.

There is, however, an additional challenge that cannot be understated.  Hardware development
is typically quite difficult for those with traditional software engineering backgrounds.
The procedural, imperative mode of software development does not lend itself naturally to
the design of hardware systems.  Furthermore, true parallelism is quite difficult to manage in
most mainstream programming languages, and the techniques for sharing state across different
parts of a software program do not lend themselves to hardware designs and implementation.

The result of these difficulties lead to the development of two frameworks for describing
hardware using the Rust Programming Language.  The first of these two frameworks, called
RustHDL is older, and has substantial development effort along with commercial products
that have been shipped and are running using firmware written with it.  However, as more
people began to use RustHDL (and people with Rust background in particular), shortcomings
became evident in its design that lead to a rewrite that is known as RHDL.  Both of
these open source projects explore the use of the Rust programming language and its
suitability for the description of hardware designs.  Although the focus is on Field Programmable
Gate Arrays (FPGAs), the work is likely applicable to other fields of design as long as
the basic principles are preserved.

The key components of both frameworks are:
\begin{itemize}
\item The use of the type system to allow for description of complex data without
  a synthesis overhead, and independent of the underlying toolchains support for
  types.
\item The use of composition and simple structs to describe hierarchies of design
  with encapsulation and the hiding of internal details.
\item The use of Rust itself as the programming language.  No new grammar is required
  and all of the tooling that comes with the Rust programming language, including training,
  tooling, and infrastructure ``just work'' with both frameworks.
\item The ability to integrate external IP cores and legacy designs (e.g., memory controllers,
  serializer/deserializers, PLLs, etc.) which cannot be described from first principles.
\item High performance simulation of hardware designs is built into the framework, so that
  testbenches and verification can be performed without the use of additional tools.
\end{itemize}

The organization of this paper is as follows.  Section~\ref{sec:related} describes related
works.  This is a particularly fruitful time for innovation in this space, and only a
sample of projects are discussed, primarily in their relationship to RustHDL and/or RHDL.
Section~\ref{sec:basics} describes the basics of how Rust is used as an HDL using the RustHDL
framework.  Section~\ref{sec:rhdl} describes how RHDL addresses the shortcomings discovered
during the development of RustHDL, with a particular focus on support for algebraic data types
and Rust syntax.  This section also touches briefly on the internals of the RHDL compiler and
how it cooperates with the Rust compiler.  Section~\ref{sec:roadmap} describes the roadmap for
the future of RHDL, and describes some of the limitations that could not be overcome.  Conclusions
and lessons learned are presented Section~\ref{sec:summary}.

\section{Background}

In considering the scope of background work, we will focus on the use of the Rust programming
language as primary in the consideration of Hardware Design Languages.  There are many modern
approaches to HDL in development (see [spade] and the references therein for a good overview).  
From a high level, however, programming languages can influence HDLs in one of two ways:

\begin{itemize}
  \item Through grammar similarity.  A number of HDLs use syntaxes that are inspired or similar to
  programming languages in order to make the transition from software to hardware design easier.
  For example, Chisel [chisel] uses a Scala-like syntax, and Spade [spade] uses a Rust-like syntax.
  The underlying implementation language is irrelevant (it happens to be Rust in the case of Spade).
  What is important is that the background experience and knowledge of the developer can be brought 
  to bear when understanding hardware descriptions if they use syntax and patterns from a broadly
  used programming language.
  \item As a host environment for the design.  In this case, a subset of the programming language (
    very similar in nature to a "synthesizable subset") is carved out of the broader programming language
    through some means, and then used to generate hardware designs within the context of the overall
    programming language.  An example here is MyHDL, which uses Python as the host language, and 
    designs are expressed in a subset of Python that is synthesizable into Verilog.
\end{itemize}

While there are several examples of HDLs that use Rust as the inspiration language, RustHDL and
RHDL fall into the second category, and the author could find no other examples like them.  In the 
first category, the most prominent examples are XLS [xls] and Spade [spade] and Veryl [veryl].  
In all three of these cases, the language is inspired by the syntax of Rust, and usually the type
system, but the actual tooling is separate from the Rust programming language and ecosystem.

RustHDL and RHDL are different in that \emph{hardware designs are expressed as valid Rust programs}.
This means that before a design can be synthesized, it must first pass the checks of the Rust
compiler.   It also means that much of the infrastructure of the Rust programming language can be reused
by hardware designers.  Things like package management, documentation, test management and IDE 
integration all come for free.

Finally, there is a rich history of using functional programming languages to describe HDL, and
with good reason.  RHDL in particular focuses on the functional aspects of Rust, and as a result 
provides some additional benefits to designers.  And both RustHDL and RHDL allow for high performance 
multithreaded simulation of the designs, all from within the same Rust ecosystem.

\section{RustHDL Basics}

RustHDL uses two aspects of the Rust programming language to describe hardware designs.  The first
is the use of traits.  An example is quite instructive, so here is an example of a simple strobe
circuit, written in RustHDL, and provided in the widget library that is published on crates.io.  Like 
every circuit desribed in RustHDL, it consists of three parts.  The first is the definition of the
circuit components, which includes input and output signals, as well as any internal components.
The use of Rust's `pub` mechanism is used to hide internal details that should not be accessed
outside of the circuit.  The struct essentially describes the architecture of the circuit, including
input and output ports, types, and internal structure.

\begin{minted}{rust}
use rust_hdl_core::prelude::*;
use crate::{dff::DFF, dff_setup};

/// A [Strobe] generates a periodic pulse train, with a single clock-cycle wide pulse
/// at the prescribed frequency.  The argument [N] of the generic [Strobe<N>] is used
/// to size the counter that stores the internal delay value.  Unfortunately, Rust const
/// generics are currently not good enough to compute [N] on the fly.  However, a compile
/// time assert ensures that the number of clock cycles between pulses does not overflow
/// the [N]-bit wide register inside the [Strobe].
#[derive(Clone, Debug, LogicBlock)]
pub struct Strobe<const N: usize> {
    /// Set this to true to enable the pulse train.
    pub enable: Signal<In, Bit>,
    /// This is the strobing signal - it will fire for 1 clock cycle such that the strobe frequency is generated.
    pub strobe: Signal<Out, Bit>,
    /// The clock that drives the [Strobe].  All signals are synchronous to this clock.
    pub clock: Signal<In, Clock>,
    threshold: Constant<Bits<N>>,
    counter: DFF<Bits<N>>,
}
\end{minted}

A couple of additional observations.  Signals have both a direction and a type.  The type of
signal can be any Rust type that implements the `Synth` trait, and can include custom user
types and structs.  The `derive` macro is used to generate the necessary boilerplate to make the 
`Strobe` struct implement the `Block` and `Logic` traits.  The details of these implementations are
unimportant, and the user can simply use the `Strobe` struct as if it were a normal Rust struct.  Also,
the D-type flip flop (DFF), which is critical to synchronous designs, is simply another circuit element that
is included in the internal structure of the `Strobe` struct.  The RustHDL DFF is parameterized or 
generic over the type of data it holds, so in this case, it is holding an N-bit wide value.  The value of
`N` is provided when the strobe is created.

The second part of the circuit encapsulates its behavior.  This is done by implementing the `Logic` trait,
and requires only a single function, called `update`:

\begin{minted}{rust}
impl<const N: usize> Logic for Strobe<N> {
  #[hdl_gen]
  fn update(&mut self) {
      // Connect the counter clock to my clock
      dff_setup!(self, clock, counter);
      if self.enable.val() {
          self.counter.d.next = self.counter.q.val() + 1;
      }
      self.strobe.next = self.enable.val() & (self.counter.q.val() == self.threshold.val());
      if self.strobe.val() {
          self.counter.d.next = 1.into();
      }
  }
}
\end{minted}

Here, the \verb|#[hdl_gen]| attribute is attached to the `update` function to provide a way to convert the function
into Verilog.  The key thing to note, however, is that \emph{without this attribute, the function is still a valid Rust function}.
This means that the `update` function can be tested, debugged, and run as a normal Rust function.  The \verb|#[hdl_gen]| adds
additional constraints to the code to ensure that it is synthesizable, but the Rust compiler still does the work of ensuring that the 
program input is correct.

The `update` function follows a few rules that are laid out in the documentation.  In short:
\begin{itemize}
  \item Signals have two endpoints.  A `.val()` endpoint that represents their current value, and a `.next` endpoint that represents
    the value that the circuit is driving them to.
  \item Several Rust flow control primitives are allowed, including `match, if` and a very limited `for`.  
  \item The ideas of blocking and non-blocking assignments, which are confusing from a software perspective are removed.  
  \item Local variables are allowed, but they must be declared in the architecture of the circuit, like any other element.
  These signals have a type fo `Local`, and are declared as \verb|my_sig: Signal<Local, T>|.
\end{itemize}

The last part of the circuit is the construction and initialization.  Here again, there is nothing special in RustHDL.  The 
constructor function is just a normal Rust function that populates the contents of the struct.  Because it can do anything
that a normal Rust function can do, arbitrary checks and computations (which are done prior to hardware synthesis) can be 
accomplished in the constructor.  For the completeness, here is the constructor for the `Strobe`, which includes some checks 
and takes frequencies as inputs:

\begin{minted}{rust}
impl<const N: usize> Strobe<N> {
  /// Generate a [Strobe] widget that can be used in a RustHDL circuit.
  ///
  /// # Arguments
  ///
  /// * `frequency`: The frequency (in Hz) of the clock signal driving the circuit.
  /// * `strobe_freq_hz`: The desired frequency in Hz of the output strobe.  Note that
  /// the strobe frequency will be rounded to something that can be obtained by dividing
  /// the input clock by an integer.  As such, it may not produce exactly the desired
  /// frequency, unless `frequency`/`strobe_freq_hz` is an integer.
  ///
  /// returns: Strobe<{ N }>
  ///
  /// # Examples
  ///
  /// See [BlinkExample] for an example.
  pub fn new(frequency: u64, strobe_freq_hz: f64) -> Self {
      let clock_duration_femto = freq_hz_to_period_femto(frequency as f64);
      let strobe_interval_femto = freq_hz_to_period_femto(strobe_freq_hz);
      let interval = strobe_interval_femto / clock_duration_femto;
      let threshold = interval.round() as u64;
      assert!((threshold as u128) < (1_u128 << (N as u128)));
      assert!(threshold > 2);
      Self {
          enable: Signal::default(),
          strobe: Signal::default(),
          clock: Signal::default(),
          threshold: Constant::new(threshold.into()),
          counter: Default::default(),
      }
  }
}
\end{minted}

Unfortunately, Rust does not currently support much in the way of computation with generic parameters such as the 
size of the internal counter `N` for the strobe.  As a result, checks are made at construction time to ensure that 
counter is wide enough to not roll over before the next strobe is generated.



\section{Ease of Use}

\subsection{Maintaining the Integrity of the Specifications}

The IEEEtran class file is used to format your paper and style the text. All margins, 
column widths, line spaces, and text fonts are prescribed; please do not 
alter them. You may note peculiarities. For example, the head margin
measures proportionately more than is customary. This measurement 
and others are deliberate, using specifications that anticipate your paper 
as one part of the entire proceedings, and not as an independent document. 
Please do not revise any of the current designations.

\section{Prepare Your Paper Before Styling}
Before you begin to format your paper, first write and save the content as a 
separate text file. Complete all content and organizational editing before 
formatting. Please note sections \ref{AA}--\ref{SCM} below for more information on 
proofreading, spelling and grammar.

Keep your text and graphic files separate until after the text has been 
formatted and styled. Do not number text heads---{\LaTeX} will do that 
for you.

\subsection{Abbreviations and Acronyms}\label{AA}
Define abbreviations and acronyms the first time they are used in the text, 
even after they have been defined in the abstract. Abbreviations such as 
IEEE, SI, MKS, CGS, ac, dc, and rms do not have to be defined. Do not use 
abbreviations in the title or heads unless they are unavoidable.

\subsection{Units}
\begin{itemize}
\item Use either SI (MKS) or CGS as primary units. (SI units are encouraged.) English units may be used as secondary units (in parentheses). An exception would be the use of English units as identifiers in trade, such as ``3.5-inch disk drive''.
\item Avoid combining SI and CGS units, such as current in amperes and magnetic field in oersteds. This often leads to confusion because equations do not balance dimensionally. If you must use mixed units, clearly state the units for each quantity that you use in an equation.
\item Do not mix complete spellings and abbreviations of units: ``Wb/m\textsuperscript{2}'' or ``webers per square meter'', not ``webers/m\textsuperscript{2}''. Spell out units when they appear in text: ``. . . a few henries'', not ``. . . a few H''.
\item Use a zero before decimal points: ``0.25'', not ``.25''. Use ``cm\textsuperscript{3}'', not ``cc''.)
\end{itemize}

\subsection{Equations}
Number equations consecutively. To make your 
equations more compact, you may use the solidus (~/~), the exp function, or 
appropriate exponents. Italicize Roman symbols for quantities and variables, 
but not Greek symbols. Use a long dash rather than a hyphen for a minus 
sign. Punctuate equations with commas or periods when they are part of a 
sentence, as in:
\begin{equation}
a+b=\gamma\label{eq}
\end{equation}

Be sure that the 
symbols in your equation have been defined before or immediately following 
the equation. Use ``\eqref{eq}'', not ``Eq.~\eqref{eq}'' or ``equation \eqref{eq}'', except at 
the beginning of a sentence: ``Equation \eqref{eq} is . . .''

\subsection{\LaTeX-Specific Advice}

Please use ``soft'' (e.g., \verb|\eqref{Eq}|) cross references instead
of ``hard'' references (e.g., \verb|(1)|). That will make it possible
to combine sections, add equations, or change the order of figures or
citations without having to go through the file line by line.

Please don't use the \verb|{eqnarray}| equation environment. Use
\verb|{align}| or \verb|{IEEEeqnarray}| instead. The \verb|{eqnarray}|
environment leaves unsightly spaces around relation symbols.

Please note that the \verb|{subequations}| environment in {\LaTeX}
will increment the main equation counter even when there are no
equation numbers displayed. If you forget that, you might write an
article in which the equation numbers skip from (17) to (20), causing
the copy editors to wonder if you've discovered a new method of
counting.

{\BibTeX} does not work by magic. It doesn't get the bibliographic
data from thin air but from .bib files. If you use {\BibTeX} to produce a
bibliography you must send the .bib files. 

{\LaTeX} can't read your mind. If you assign the same label to a
subsubsection and a table, you might find that Table I has been cross
referenced as Table IV-B3. 

{\LaTeX} does not have precognitive abilities. If you put a
\verb|\label| command before the command that updates the counter it's
supposed to be using, the label will pick up the last counter to be
cross referenced instead. In particular, a \verb|\label| command
should not go before the caption of a figure or a table.

Do not use \verb|\nonumber| inside the \verb|{array}| environment. It
will not stop equation numbers inside \verb|{array}| (there won't be
any anyway) and it might stop a wanted equation number in the
surrounding equation.

\subsection{Some Common Mistakes}\label{SCM}
\begin{itemize}
\item The word ``data'' is plural, not singular.
\item The subscript for the permeability of vacuum $\mu_{0}$, and other common scientific constants, is zero with subscript formatting, not a lowercase letter ``o''.
\item In American English, commas, semicolons, periods, question and exclamation marks are located within quotation marks only when a complete thought or name is cited, such as a title or full quotation. When quotation marks are used, instead of a bold or italic typeface, to highlight a word or phrase, punctuation should appear outside of the quotation marks. A parenthetical phrase or statement at the end of a sentence is punctuated outside of the closing parenthesis (like this). (A parenthetical sentence is punctuated within the parentheses.)
\item A graph within a graph is an ``inset'', not an ``insert''. The word alternatively is preferred to the word ``alternately'' (unless you really mean something that alternates).
\item Do not use the word ``essentially'' to mean ``approximately'' or ``effectively''.
\item In your paper title, if the words ``that uses'' can accurately replace the word ``using'', capitalize the ``u''; if not, keep using lower-cased.
\item Be aware of the different meanings of the homophones ``affect'' and ``effect'', ``complement'' and ``compliment'', ``discreet'' and ``discrete'', ``principal'' and ``principle''.
\item Do not confuse ``imply'' and ``infer''.
\item The prefix ``non'' is not a word; it should be joined to the word it modifies, usually without a hyphen.
\item There is no period after the ``et'' in the Latin abbreviation ``et al.''.
\item The abbreviation ``i.e.'' means ``that is'', and the abbreviation ``e.g.'' means ``for example''.
\end{itemize}
An excellent style manual for science writers is \cite{b7}.

\subsection{Authors and Affiliations}
\textbf{The class file is designed for, but not limited to, six authors.} A 
minimum of one author is required for all conference articles. Author names 
should be listed starting from left to right and then moving down to the 
next line. This is the author sequence that will be used in future citations 
and by indexing services. Names should not be listed in columns nor group by 
affiliation. Please keep your affiliations as succinct as possible (for 
example, do not differentiate among departments of the same organization).

\subsection{Identify the Headings}
Headings, or heads, are organizational devices that guide the reader through 
your paper. There are two types: component heads and text heads.

Component heads identify the different components of your paper and are not 
topically subordinate to each other. Examples include Acknowledgments and 
References and, for these, the correct style to use is ``Heading 5''. Use 
``figure caption'' for your Figure captions, and ``table head'' for your 
table title. Run-in heads, such as ``Abstract'', will require you to apply a 
style (in this case, italic) in addition to the style provided by the drop 
down menu to differentiate the head from the text.

Text heads organize the topics on a relational, hierarchical basis. For 
example, the paper title is the primary text head because all subsequent 
material relates and elaborates on this one topic. If there are two or more 
sub-topics, the next level head (uppercase Roman numerals) should be used 
and, conversely, if there are not at least two sub-topics, then no subheads 
should be introduced.

\subsection{Figures and Tables}
\paragraph{Positioning Figures and Tables} Place figures and tables at the top and 
bottom of columns. Avoid placing them in the middle of columns. Large 
figures and tables may span across both columns. Figure captions should be 
below the figures; table heads should appear above the tables. Insert 
figures and tables after they are cited in the text. Use the abbreviation 
``Fig.~\ref{fig}'', even at the beginning of a sentence.

\begin{table}[htbp]
\caption{Table Type Styles}
\begin{center}
\begin{tabular}{|c|c|c|c|}
\hline
\textbf{Table}&\multicolumn{3}{|c|}{\textbf{Table Column Head}} \\
\cline{2-4} 
\textbf{Head} & \textbf{\textit{Table column subhead}}& \textbf{\textit{Subhead}}& \textbf{\textit{Subhead}} \\
\hline
copy& More table copy$^{\mathrm{a}}$& &  \\
\hline
\multicolumn{4}{l}{$^{\mathrm{a}}$Sample of a Table footnote.}
\end{tabular}
\label{tab1}
\end{center}
\end{table}

\begin{figure}[htbp]
\centerline{\includegraphics{fig1.png}}
\caption{Example of a figure caption.}
\label{fig}
\end{figure}

Figure Labels: Use 8 point Times New Roman for Figure labels. Use words 
rather than symbols or abbreviations when writing Figure axis labels to 
avoid confusing the reader. As an example, write the quantity 
``Magnetization'', or ``Magnetization, M'', not just ``M''. If including 
units in the label, present them within parentheses. Do not label axes only 
with units. In the example, write ``Magnetization (A/m)'' or ``Magnetization 
\{A[m(1)]\}'', not just ``A/m''. Do not label axes with a ratio of 
quantities and units. For example, write ``Temperature (K)'', not 
``Temperature/K''.

\section*{Acknowledgment}

The preferred spelling of the word ``acknowledgment'' in America is without 
an ``e'' after the ``g''. Avoid the stilted expression ``one of us (R. B. 
G.) thanks $\ldots$''. Instead, try ``R. B. G. thanks$\ldots$''. Put sponsor 
acknowledgments in the unnumbered footnote on the first page.

\section*{References}

Please number citations consecutively within brackets \cite{b1}. The 
sentence punctuation follows the bracket \cite{b2}. Refer simply to the reference 
number, as in \cite{b3}---do not use ``Ref. \cite{b3}'' or ``reference \cite{b3}'' except at 
the beginning of a sentence: ``Reference \cite{b3} was the first $\ldots$''

Number footnotes separately in superscripts. Place the actual footnote at 
the bottom of the column in which it was cited. Do not put footnotes in the 
abstract or reference list. Use letters for table footnotes.

Unless there are six authors or more give all authors' names; do not use 
``et al.''. Papers that have not been published, even if they have been 
submitted for publication, should be cited as ``unpublished'' \cite{b4}. Papers 
that have been accepted for publication should be cited as ``in press'' \cite{b5}. 
Capitalize only the first word in a paper title, except for proper nouns and 
element symbols.

For papers published in translation journals, please give the English 
citation first, followed by the original foreign-language citation \cite{b6}.

\begin{thebibliography}{00}
\bibitem{b1} G. Eason, B. Noble, and I. N. Sneddon, ``On certain integrals of Lipschitz-Hankel type involving products of Bessel functions,'' Phil. Trans. Roy. Soc. London, vol. A247, pp. 529--551, April 1955.
\bibitem{b2} J. Clerk Maxwell, A Treatise on Electricity and Magnetism, 3rd ed., vol. 2. Oxford: Clarendon, 1892, pp.68--73.
\bibitem{b3} I. S. Jacobs and C. P. Bean, ``Fine particles, thin films and exchange anisotropy,'' in Magnetism, vol. III, G. T. Rado and H. Suhl, Eds. New York: Academic, 1963, pp. 271--350.
\bibitem{b4} K. Elissa, ``Title of paper if known,'' unpublished.
\bibitem{b5} R. Nicole, ``Title of paper with only first word capitalized,'' J. Name Stand. Abbrev., in press.
\bibitem{b6} Y. Yorozu, M. Hirano, K. Oka, and Y. Tagawa, ``Electron spectroscopy studies on magneto-optical media and plastic substrate interface,'' IEEE Transl. J. Magn. Japan, vol. 2, pp. 740--741, August 1987 [Digests 9th Annual Conf. Magnetics Japan, p. 301, 1982].
\bibitem{b7} M. Young, The Technical Writer's Handbook. Mill Valley, CA: University Science, 1989.
\end{thebibliography}
\vspace{12pt}
\color{red}
IEEE conference templates contain guidance text for composing and formatting conference papers. Please ensure that all template text is removed from your conference paper prior to submission to the conference. Failure to remove the template text from your paper may result in your paper not being published.

\end{document}
