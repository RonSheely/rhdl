\documentclass[sigplan,screen,sigconf]{acmart}
\usepackage{tcolorbox}
\usepackage{minted}
\settopmatter{printfolios=false,printacmref=false}
\bibliographystyle{ACM-Reference-Format}

\setcopyright{rightsretained}
\acmDOI{}
\acmISBN{}
\acmConference[LATTE '24]{3nd Workshop on Languages, Tools, and Techniques for Accelerator Design}{March 26, 2023}{Vancouver, BC, Canada}

\author{Samit Basu}
\affiliation{
  basu.samit@gmail.com
  \country{Fremont CA, USA}
}

\begin{document}

\title{Rust as a Hardware Description Language}

\begin{abstract}
Rust \cite{b9} makes an excellent language for hardware description.  A number of new HDLs are 
Rust-inspired in syntax \cite{b1},\cite{b4},
but RustHDL\cite{b6} is actually a framework for turning ordinary Rust code into firmware.  The first attempt,
while successful had several shortcomings that are discussed in this paper.  The new framework, RHDL\cite{b10},
should address those shortcomings and significantly ease the use of Rust for hardware design.
\end{abstract}

\maketitle

\section{Introduction}

While the field of HDLs may be crowded, I propose the use of the Rust Programming Language (RPL)
as a hardware description language.  Beyond its status as most loved of the programming languages \cite{b0},
Rust has steadily been gaining traction as a serious language for systems programming, embedded 
software, and other mission critical applications.  The particular features of Rust that are 
relevant to hardware description include:
\begin{itemize}
\item Static typing and sane syntax
\item Functional programming features
\item A powerful macro system
\item Package management and a growing open-source ecosystem
\item Built-in test capabilities
\item Significant tooling and infrastructure support
\item Generics and const generics
\end{itemize}
The RHDL framework and its predecessor RustHDL took advantage of all of these features to
create an environment for hardware description that is powerful, easy to use, extensible and
open.  The end goal is to enable a wider class of engineers to develop high quality hardware
by reusing their skills as Rust developers in the hardware domain.  This paper briefly describes
the RustHDL approach to developing FPGA firmware using the RPL, and then identifies the
observed shortcomings and how they may be addressed in the upcoming RHDL framework.  
Finally, I touch upon some of the unsolved problems in using Rust for hardware description.

\section{Syntax}
RustHDL is not a new language.  Instead it is a set of libraries and macros along with a 
subset of the Rust programming language that can be used to generate firmware.  The key
principle of RustHDL is:

\begin{tcolorbox}
RustHDL designs are valid Rust programs that can be compiled and run on a host computer
using the included event-based simulator.
\end{tcolorbox}

In this sense, RustHDL is embedded in the Rust language much as MyHDL is embedded in Python \cite{b3},
and Chisel is embedded in Scala \cite{b2}.  The immediate implication is that:
\begin{itemize}
  \item All RustHDL designs must pass the strenuous correctness checks of the Rust compiler \verb|rustc|.
  \item An entire class of potential bugs are eliminated, such as type mismatches, use-before-initialization,
  unassigned outputs, etc.
  \item Tools such as \verb|clippy| and \verb|rust-analyzer| can immediately be used to
  check, lint and analyze code with no additional investment.
  \item The Rust test framework can be used to test the designs directly.
\end{itemize}

The syntax should be fairly familiar to anyone comfortable with Rust.  The following is an example
of a simple SPI master in RustHDL, generic over the transaction size \verb|N|, edited for brevity:

\begin{minted}[fontsize=\footnotesize]{rust}
  #[derive(LogicBlock)]
  pub struct SPIMaster<const N: usize> {
      pub clock: Signal<In, Clock>,
      pub data_outbound: Signal<In, Bits<N>>,
      pub start_send: Signal<In, Bit>,
      pub data_inbound: Signal<Out, Bits<N>>,
      pub wires: SPIWiresMaster,
      local_signal: Signal<Local, Bit>,
      state: DFF<SPIState>,
      cs_off: Constant<Bit>,
  }
\end{minted}  

The \verb|pub| keyword is used to denote the visibility of the signals.  Signals 
marked with a direction, and type.  Internal components such as flip flops and strobes
are all included in the top level struct, which is initialized using normal Rust code.
The \verb|Local| signal represents a local variable used in the update function, but
not otherwise exposed.  As RustHDL had no type inference, it requires explicit allocation
and types for all local variables.  The member \verb|cs_off| (along with others omitted) is a  
constant constructed at runtime that encodes the SPI mode of the bus.  Finally, 
the \verb|SPIWiresMaster| is a struct that describes the interface to the actual SPI bus.
Interfaces (unlike structs in SystemVerilog, for example) include both input and output
signals, and can be used to ``connect'' complex components with a single line.  

As an example, an interface to an SDRAM chip with a \verb|D-bit| data bus and a 13 bit 
address bus is defined as:

\begin{minted}[fontsize=\footnotesize]{rust}
  #[derive(LogicInterface, Clone, Debug, Default)]
  #[join = "SDRAMDriver"]
  pub struct SDRAMDevice<const D: usize> {
      pub clk: Signal<In, Clock>,
      pub we_not: Signal<In, Bit>,
      pub cas_not: Signal<In, Bit>,
      pub ras_not: Signal<In, Bit>,
      pub address: Signal<In, Bits<13>>,
      pub write_data: Signal<In, Bits<D>>,
      pub read_data: Signal<Out, Bits<D>>,
      pub write_enable: Signal<In, Bit>,
  }
\end{minted}

and can be connected to the corresponding signals in another IP block with a single \verb|join|
statement.  This significantly reduces the amount of error-prone wiring that must be done
by code or graphically.

Back to the SPI controller example, an update function calculates the next value of the 
signals (external and internal) based on the current state stored in the DFF \verb|state|,
which itself is a C-style enum.  Rust ensures that the state machine match/case is exhaustive. 

\section{Mental Model}
RustHDL attempts to build on HDLs like Lucid\cite{b11} to provide a more understandable mental model for how 
hardware works.  In an imperfect implementation, RustHDL defines a \verb|Signal| struct that has a 
read only endpoint \verb|x.val()| for signal \verb|x|, and a write endpoint \verb|x.next|.  

\begin{minted}[fontsize=\footnotesize]{rust}
// Design is parametric over N - the size of the counter
impl<const N: usize> Logic for Strobe<N> {
  #[hdl_gen]
  fn update(&mut self) {
    if self.enable.val() {
      self.counter.d.next = self.counter.q.val() + 1;
    }
    self.strobe.next = self.enable.val() & 
      (self.counter.q.val() == self.threshold.val());
    if self.strobe.val() {
      self.counter.d.next = 1.into();
    }
  }
} 
\end{minted}

Rust lacks write-only semantics, so the framework checks for read-before-write on the \verb|x.next| 
endpoint.  Using this nomenclature, the idea of non-blocking assignments is replaced with a
 conditional model - i.e., given the current value in the set of signals, what next value do I 
want them to take?  This mental model is coupled with analysis passes that look (with the aid 
of yosys\cite{b12} in RustHDL) for latch inferences due to missing assignments and other such issues.

\section{Simulation}
Testing of designs in RustHDL does not require the use of third party tools or tooling.  
Tests utilize a built-in event-based simulation engine that can simulate any RustHDL design.
Black box IP cores can be simulated by providing Rust equivalents of the hardware descriptions.
The simplest example is something like a block RAM, which can be trivially instantiated in 
Verilog, but requires a behavioral model in RustHDL.  In RustHDL that behavioral model is 
written in Rust, and can be substituted into the simulation environment.  Other black box
IP cores can be equivalently simulated in Rust.

Speed is a critical factor in simulation.  RustHDL is a reasonably fast simulator, 
and the Rust test framework is inherently parallel, and can run multiple tests in parallel. 
Using system calls/shell-outs, the entire synthesis and bitstream generation process 
can be handled inside the Rust ecosystem.

\section{Reuse}
Hardware designs in RustHDL are simply \verb|struct|s, and are composed of other 
hardware components via composition.  This allows for easy reuse of components, the
construction of complex designs out of simpler, smaller components, combined with sane rules of
scoping and encapsulation.  Furthermore, each of the subcomponents can be tested in 
isolation, and then tested after composition in the larger design. 

Rust is a very composable language, and \verb|crates.io| provides a natural mechanism
for sharing and reusing components.  As an example, in RustHDL, handling of hardware specific
details (such as synthesis tools and constraints files for specific FPGAs and boards) is 
handled through a \emph{board support package}.  This is simple a library that provides the
defaults, pin-outs, and other mapping information needed to generate a bitstream for a given piece of
hardware.  As an open-ended and potentially unbounded problem, the BSP can be published as
a \verb|crate| (package) in the Rust ecosystem by contributors \cite{b7}.  This decentralizes control
over one of the more challenging parts of maintaining support for a bewildering array of 
devices. 

\section{Shortcomings and the Future}
RustHDL has been used for non-trivial commercial firmware development and is deployed.  It has also seen 
some level of interest and adoption from the open source community.  Feedback from early users 
lead to the following list of shortcomings:

\begin{itemize}
  \item The subset of Rust supported by RustHDL (which is the subset of the language that can be 
  directly translated into Verilog) is too small to write ``natural'' Rust code. 
  \item RustHDL does not support Algebraic Data Types (data-carrying enums).
  \item Local variables and type inference are critical to writing clean and
  idiomatic Rust code.  
  \item Composition of functions/behavior is not possible. 
  \item Writing test-benches required an understanding of the simulator mechanics.
  \item Backends are needed for more than just Verilog.
\end{itemize}

Solving all of these problems essentially necessitated a rewrite of RustHDL.  The new framework,
called RHDL (Rust Hardware Description Language) is currently under development.  The primary
technical difference to RustHDL is the introduction of an auxiliary compiler into the processing.
This compiler works in tandem with \verb|rustc| to convert an AST of the code into a RTL-like
HDL, and then transform and optimize that representation into a form that can be synthesized.  The 
compiler is key to support of things like early returns, match and if expressions (as opposed to statements),
and other Rust-isms that are not common in HDLs, but are common in Rust.  The compiler also provides
ADT support with control over the layout of the data, and easy composition of data types into structs
of arbitrary complexity.  

\section{Conclusions}
I believe Rust is a promising basis for a hardware description language.  It offers many powerful tools
that can be utilized to build composable, reusable and correct hardware designs.  The RustHDL framework
was a first step in this direction, and the in-development RHDL framework promises to address many of the
shortcomings of the first attempt. I look forward to presenting more details on RHDL as it evolves.

\begin{thebibliography}{00}
  \bibitem{b9} ``Rust - A language empowering everyone to build reliable and efficient software'', \url{https://rust-lang.org} (Accessed Feb 1, 2024).
  \bibitem{b1} F. Skarman and O. Gustafsson, ``Spade: An Expression-Based HDL With Pipelines'', Open Source Design Automation Conference, 2023.
  \bibitem{b4} ``XLS: Accelerated HW Synthesis'', \url{https://google.github.io/xls/} (Accessed Feb 1, 2024).
  \bibitem{b6} ``RustHDL - Write FPGA Firmware using Rust!'', \url{https://rust-hdl.org/} (Accessed Feb 1, 2024).
  \bibitem{b10} ``RHDL - Rust Hardware Description Language'', \url{https://github.com/samitbasu/rhdl} (Accessed Feb 1, 2024).
  \bibitem{b0} ``Stack Overflow Developer Survey 2023'', \url{https://insights.stackoverflow.com/survey/2023} (Accessed Feb 1, 2024).
  \bibitem{b3} ``MyHDL - From Python to Silicon!'', \url{https://www.myhdl.org/} (Accessed Feb 1, 2024).
  \bibitem{b2} ``Chisel - Software-defined hardware'', \url{https://www.chisel-lang.org/} (Accessed Feb 1, 2024).
  \bibitem{b11} J. Rajewski, ``Lucid - FPGA Tutorials'', \url{https://alchitry.com/lucid/} (Accessed Feb 18, 2024).
  \bibitem{b12} C. Wolf, ``Yosys Open SYnthesis Suite'', \url{https://yosyshq.net/yosys/} (Accessed Feb 18, 2024).
  \bibitem{b7} ``rust-hdl-bsp-step-mxo2-lpc - rust-hdl board support package for STEP-MXO2-LPC'', \url{https://crates.io/crates/rust-hdl-bsp-step-mxo2-lpc} (Accessed Feb 4, 2024).
\end{thebibliography}
\end{document}
